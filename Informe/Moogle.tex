\documentclass[a4paper,12pt]{article}
\usepackage[utf8]{inputenc}
\usepackage[T1]{fontenc}
\usepackage{amsfonts}
\usepackage{graphicx}
%\usepackage[spanish]{babel}
\title{Proyecto Intrasemestral de Programaci\'on\\Moogle!}
\author{Christopher Guerra Herrero}
\date{Septiembre, 2022}
\pagestyle{headings}


\begin{document}

    \begin{figure}
        \begin{center}
            \includegraphics[scale=0.30]{Moogle.png}
        \end{center}
    \end{figure}

    \begin {titlepage}
        \centering
        {\bfseries\LARGE Universidad de la Habana \par}
        \vspace{1cm}
        {\scshape\Large Facultad de Matem\'atica y Computaci\'on \par}
        \vspace{3cm}
        {\itshape\Large Proyecto Intrasemestral de Programaci\'on\\ Moogle! \par}
        \vfill
        {\Large Autor: \par}
        {\Large Christopher Guerra Herrero \par}
        {\Large Septiembre, 2022}
    \end {titlepage}


    \newpage
    \section{Informaci\'on b\'asica}
        \subsection{Google, pero con M}
            Moogle es un {\it buscador} creado con C\# para el back-end y Razor para la
            interfaz. Su objetivo es, dado una {\it consulta} y un conjunto de ficheros
            .txt, indicarle al usuario que documentos se asemejan m\'as a su b\'usqueda.\\
                Con en el fin de hacer m\'as grata la experiencia del usuario, nuestro programa
            cuenta con una serie de herramientas para hacer m\'as eficiente la b\'usqueda (operadores),
             y sugiere consultas semejantes a la hecha por el usuario, cuando esta no tiene resultados.
            Para alcanzar estas metas, el programa hace uso de diferentes algoritmos que se
            describir\'an en los apartados siguientes
        \subsection{C\'omo funciona la b\'usqueda?}
            Para brindar resultados, Moogle, hace uso de un {\it modelo vectorial}, el cual 
            consiste en considerar como vectores n-dimensionales a los archivos .txt y a la consulta;
            y luego hallar el coseno del \'angulo formado entre estos con el fin de hallar los
            vectores con menor \'angulo, es decir, los vectores que m\'as  se aproximan.\\
                Para esta abstracci\'on de considerar un texto como un vector n-dimensional se usa
            el principio TF-IDF. Si el lector desea saber m\'as sobre este principio consulte \cite{TF-IDF}.
        \subsection{Se pueden usar chirimbolos!}
            Con el fin de hacer m\'as certera las consulta del usuario, se pueden utilizar los siguientes operadores:
            \begin{itemize}
                \item {\bf Operador \^\ :} Se usa antepuesto inmediatamente a una palabra e indica que todos los documentos brindados
                deben contener dicha palabra (Ejemplo: \^\ Garfield).
                \item {\bf Operador ! :} Se usa antepuesto inmediatamente a una palabra e indica que ninguno de los documentos brindados
                deben contener dicha palabra (Ejemplo: !Garfield).
                \item {\bf Operador * :} Se usa antepuesto inmediatamente a una palabra e indica que esa palabra es importante en la
                b\'usqueda. Es decir, si un documento contiene esta palabra su Score quedar\'a multiplicado x2. Su efecto es aditivo, es decir, pueden ponerse varios para aumentar a\'un m\'as la importancia
                de la palabra  (Ejemplo: ***Garfield).
                \item {\bf Operador \~\ :} Se usa entre dos palabras separado por espacios. Indica que entre m\'as cercanas sean estas palabras en
                un documento m\'as importante ser\'a este, puesto que el Score quedar\'a multiplicado por un coeficiente entre 1 y 10. (Ejemplo: Garfield \~\ Odi).
                \item {\bf Operador " :} Se usan para seleccionar una frase, uno al principio y otro al final. Indica que dicha frase debe aparecer textual 
                en todos los resultados de la b\'usqueda (Ejemplo: "Garfield ama a la laza\~na").
                \item {\bf Operador ' :} Se usan para seleccionar una frase, uno al principio y otro al final. Indica que dicha frase no puede aparecer textual 
                en ninguno de los resultados de la b\'usqueda (Ejemplo: 'Garfield odia a Odi').
            \end{itemize}
        \subsection{Puede que hayas querido decir...}
            Tener faltas de ortograf\'ia no nos va a impedir trabajar con Moogle ya que este est\'a dise\~nado para ofrecer {\it b\'usquedas alternativas}
            cuando la hecha por el usuario no tiene resultados(ya sea por errores en la escrituta o por la no aparici\'on de coincidencias en nuestros documentos).
            Esta sugerencia aparecer\'a en la parte inferior del caj\'on de b\'usqueda con color de fuente azul(palabra caliente).
    \section{Sobre la implementaci\'on}
        Nuestro c\'odigo trabaja con tres clases: Moogle, la clase principal; Metodos, clase para agrupar los diferentes m\'etodos que se necesitan en Moogle; y Ficha,
         clase que caracteriza a los diferentes documentos y es usada en todo el c\'odigo indistintamente. La clase Moogle cuenta con dos m\'etodos , el constructor y Query().
         Al cargar nuestra p\'agina principal en el navegador una instancia de Moogle es creada, por lo cual se ejecuta el constructor de la clase y se cargan los documentos(enti\'endase
         aplicar el principio del TF-IDF a los documentos). Para ello se hace uso de diferentes m\'etodos de la clase Metodos y almacenando las informaciones en diferenres instancias de ficha.
        \subsection{Flujo}
            \subsubsection{Constructor Moogle()}
                El constructor de la clase Moogle se encarga de cargar los archivos .txt y calcularles su TF-IDF. En este proceso son utilizados dos m\'etodos est\'aticos de la clase Metodos. 
                Adem\'as de guardar los resultados obtenidos en la propiedad TFIDF de las instancias de ficha, tambi\'en son almacenadas las posiciones de cada palabra del documento en un diccionario
                que asocia cada string con una lista de enteros(esto tambi\'en constituye una propiedad de ficha), para luego facilitar la implementaci\'on del operador \~\ .
            \subsection{M\'etodo Query}
                El m\'etodo Query comienza inicializando en 1 la propiedad Score de todos los documentos(fichas) que tenemos. La propiedad Score 
                de la clase ficha, contiene un n\'umero que representa la importancia de ese documento para una determinada b\'usqueda. Por defecto esta propiedad
                es inicializada en 1 y sus incrementos se realizan con el producto usual, es decir: al calcular el coseno entre el vector que representa
                a la query y el que representa a un documento, este valor se le multiplica a la propiedad Score; al usar determinados operadores que deben incrementar el Score
                por un coeficiente determinado, simplemente se multiplica el Score por ese valor. La importancia principal de este procedimiento reside
                en que en determinadas circunstancias es necesario anular algunos Score e independientemente de lo que ocurra luego, estos no incrementan
                m\'as; al multiplicar un Score por cero se logra lo deseado.\\

                A continuaci\'on se procede a calcular el TFIDF de la consulta haciendo uso de otro m\'etodo de la clase Metodos. En este punto son implementados los operadores de b\'usqueda tambi\'en,
                los cuales afectaran directamente a la propiedad Score de los documentos.
                Nuestro programa le mostrar\'a al usuario con los resultados un peque\~no fragmento de su texto en el que aparezca la palabra con mayor TFIDF de la consulta que contenga el documento. 
                Por esta raz\'on, cuando se calcula el TFIDF de las palabras en la consulta, tambi\'en se guardan un orden de estas de acuerdo a su importancia.\\
                Ahora estamos listos para hallar el coseno entre el vector que representa la consulta y los que representan a los diferentes documentos. Para esto se llama al m\'etodo HallaCoseno()
                de la clase Metodos y guardamos estos valores multiplic\'andoselos al Score de cada documento. Bajo este criterio ordenaremos los documentos de tal forma que ser\'a
                {\it mejor} aquel con mayor Score.\\

                Para seleccionar el fragmento de texto que acompa\~nar\'a a los resultados, escogemos la palabra m\'as importante de la consulta(con mayor TF-IDF) que este documento contenga y seleccionamos 300
                caracteres despu\'es de la primera aparici\'on de esta palabra.\\

                Con los documentos m\'as relevantes para una determinada consulta, Snippet para ellos y sus Score se realizan diferentes instancias de la clase SearchItem.\\
                Para lograr las sugerencias mencionadas anteriormente cuando no hay resultados para la consulta se aplica el algoritmo de Levenshtein. Para mayor comprensi\'on consulte \cite{Levenshtein}.\\

                Con las instancias de SearchItem y las sugerencias, instanciamos un objeto de la clase SearchResult y esta es la que devuelve el m\'etodo Query, dando as\'i los resultados buscados.



    \newpage
    \begin{thebibliography}{5}
        \bibitem{TF-IDF} https://es.m.wikipiedia.org/wiki/Tf-idf
        \bibitem{Levenshtein} https://es.m.wikipiedia.org/wiki/Distancia\_de\_Levenshtein
    \end{thebibliography}
\end{document}